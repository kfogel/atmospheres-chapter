%%%%%%%%%%%%%%%%%%%% author.tex %%%%%%%%%%%%%%%%%%%%%%%%%%%%%%%%%%%
%
% template for chapters to the Handbook of Exoplanets
% modified by H. Deeg from the 'template.tex' provided by Springer for the svmult.cls class
% 20Mar 2016
%
%%%%%%%%%%%%%%%% Springer %%%%%%%%%%%%%%%%%%%%%%%%%%%%%%%%%%


% RECOMMENDED %%%%%%%%%%%%%%%%%%%%%%%%%%%%%%%%%%%%%%%%%%%%%%%%%%%
\documentclass[graybox,natbib,nosecnum]{svmult}
\bibpunct{(}{)}{;}{a}{}{,} % suppress commas between author-names and year

\pdfoutput=1   %forces use of pdflatex. Disable if you prefer to use .eps or .ps figures.
% choose options for [] as required from the list
% in the Reference Guide

\usepackage{mathptmx}       % selects Times Roman as basic font
\usepackage{helvet}         % selects Helvetica as sans-serif font
\usepackage{courier}        % selects Courier as typewriter font
\usepackage{type1cm}        % activate if the above 3 fonts are
                            % not available on your system

\usepackage{makeidx}         % allows index generation
\usepackage{graphicx}        % standard LaTeX graphics tool
                             % when including figure files
\usepackage{multicol}        % used for the two-column index
\usepackage[bottom]{footmisc}% places footnotes at page bottom
\usepackage[normalem]{ulem}	% for strike-through of text with \sout{}  
\usepackage{hyperref}  %for hyperlinks

\usepackage{soul}   % for high-lighting of text
% see the list of further useful packages
% in the Reference Guide

% expansions of  journal abbreviations from bibtex entries by ADS
% adapted to Springer Basic style (no periods in abbreviations)
\newcommand*\aap{A\&A}
\let\astap=\aap
\newcommand*\aapr{A\&A Rev}
\newcommand*\aaps{A\&AS}
\newcommand*\actaa{Acta Astron}
\newcommand*\aj{AJ}
\newcommand*\ao{Appl Opt}
\let\applopt\ao
\newcommand*\apj{ApJ}
\newcommand*\apjl{ApJ}
\let\apjlett\apjl
\newcommand*\apjs{ApJS}
\let\apjsupp\apjs
\newcommand*\aplett{Astrophys Lett}
\newcommand*\apspr{Astrophys Space Phys Res}
\newcommand*\apss{Ap\&SS}
\newcommand*\araa{ARA\&A}
\newcommand*\azh{AZh}
\newcommand*\baas{BAAS}
\newcommand*\bac{Bull astr Inst Czechosl}
\newcommand*\bain{Bull Astron Inst Netherlands}
\newcommand*\caa{Chinese Astron Astrophys}
\newcommand*\cjaa{Chinese J Astron Astrophys}
\newcommand*\fcp{Fund Cosmic Phys}
\newcommand*\gca{Geochim Cosmochim Acta}
\newcommand*\grl{Geophys Res Lett}
\newcommand*\iaucirc{IAU Circ}
\newcommand*\icarus{Icarus}
\newcommand*\jcap{J Cosmology Astropart Phys}
\newcommand*\jcp{J Chem Phys}
\newcommand*\jgr{J Geophys Res}
\newcommand*\jqsrt{J Quant Spectr Rad Transf}
\newcommand*\jrasc{JRASC}
\newcommand*\memras{MmRAS}
\newcommand*\memsai{Mem Soc Astron Italiana}
\newcommand*\mnras{MNRAS}
\newcommand*\na{New A}
\newcommand*\nar{New A Rev}
\newcommand*\nat{Nature}
\newcommand*\nphysa{Nucl Phys A}
\newcommand*\pasa{PASA}
\newcommand*\pasj{PASJ}
\newcommand*\pasp{PASP}
\newcommand*\physrep{Phys Rep}
\newcommand*\physscr{Phys Scr}
\newcommand*\planss{Planet Space Sci}
\newcommand*\pra{Phys Rev A}
\newcommand*\prb{Phys Rev B}
\newcommand*\prc{Phys Rev C}
\newcommand*\prd{Phys Rev D}
\newcommand*\pre{Phys Rev E}
\newcommand*\prl{Phys Rev Lett}
\newcommand*\procspie{Proc SPIE}
\newcommand*\qjras{QJRAS}
\newcommand*\rmxaa{Rev Mexicana Astron Astrofis}
\newcommand*\skytel{S\&T}
\newcommand*\solphys{Sol Phys}
\newcommand*\sovast{Soviet Ast}
\newcommand*\ssr{Space Sci Rev}
\newcommand*\zap{ZAp}


\newcommand{\hbindex}[1]{\hl{#1}\index{#1}}  %highlights index entries

\makeindex             % used for the subject index
                       % please use the style svind.ist with
                       % your makeindex program

%%%%%%%%%%%%%%%%%%%%%%%%%%%%%%%%%%%%%%%%%%%%%%%%%%%%%%%%%%%%%%%%%%%%%%%%%%%%%%%%%%%%%%%%%

\begin{document}

\title*{Exoplanet Atmosphere Measurements from Transmission Spectroscopy and other Planet-Star Combined Light Observations}
% Use \titlerunning{Short Title} for an abbreviated version of
% your contribution title if the original one is too long
\author{Laura Kreidberg}
% Use \authorrunning{Short Title} for an abbreviated version of
% your contribution title if the original one is too long
\institute{Laura Kreidberg \at Harvard Society of Fellows, 78 Mount Auburn Street, Cambridge, MA, 02138, \email{laura.kreidberg@cfa.harvard.edu}}
%
% Use the package "url.sty" to avoid
% problems with special characters
% used in your e-mail or web address
%
\maketitle


\abstract{Abstract.}
%This document is intended as a template and guide for the preparation of chapters for the Handbook of Exoplanets, using latex. It complements the `Guidelines for Authors of the Exoplanet Handbook', which are distributed as a separate document. This template was adapted from Springer's template `author.tex' for contributed books, using the style svmult.cls (distributed together with this template). \\ Each chapter should be preceded by an abstract (10--15 lines long) that summarizes the content. The abstract will appear \textit{online} at \url{www.SpringerLink.com} and at ADS and be available with unrestricted access. This allows unregistered users to read the abstract as a teaser for the complete chapter. For the title you are encouraged to follow the tips for Search Engine Optimization given below.}


\section{Introduction }
%"Don't you just hate it when you travel 1,200 light years to a planet scientists have assured you is "Earthlike," and you get there, and there's, like, NO ATMOSPHERE?!!" -- Martin BG, Potomac, MD



\section{Techniques} 
In this chapter we will focus on exoplanet atmosphere characterization using the combined light from the planet and its host star. In contrast to direct imaging, which seeks to mask the stellar flux, the combined light approach capitalizes on the star as a constant reference point. 

%As the planet moves through its orbit, it can absorb
%Changes in the sum total brightness with time can be traced to emission, reflection, or absorption by the planet's atmosphere.

The most widely used combined-light technique is transmission spectroscopy. For this method, the planet is observed in transit as it passes in front of its host. During a transit, the planet blocks a small fraction of the stellar flux that is roughly equal to the area of the planet relative to the area of the host star. We refer to this fraction as the \hbindex{transit depth}, $\delta = R_p^2/R_s^2$.

The key idea behind transmission spectroscopy is that the apparent size of the planet is not actually a single value: rather, it depends on the wavelength of light used for the observation.  This change in transit depth with wavelength, referred to as the \hbindex{transmission spectrum}, is due to the planet's atmosphere. At wavelengths where the atmosphere is relatively more opaque, the planet blocks slightly more stellar flux.

The typical scale of features in the transmission spectrum is set by the \hbindex{atmospheric scale height} $H$. For an atmosphere in hydrostatic equilibrium, the scale height is 

\begin{equation}
H = \frac{K_bT_{eq}}{\mu g}
\end{equation}
where $K_b$ is the Boltzmann constant, $T_{eq}$ is the planet's \hbindex{equilibrium temperature}, $\mu$ is the mean molecular mass, and $g$ is the surface gravity (FIXME). A hydrogen-dominated atmosphere like that of Jupiter has $\mu = 2.2$, whereas an Earth-like atmosphere has $\mu = 29$ (FIXME).

Assuming the planet's radius is larger by $n$ scale heights at wavelengths with high opacity, the amplitude of features in the spectrum is given by

\begin{eqnarray}
\delta_\lambda &=& \frac{(R_p + nH_\lambda)^2}{R_s^2} - \frac{R_p^2}{R_s^2}\\
 & \approx & 2R_pH_\lambda/R_s^2.
\end{eqnarray} 

Observations of transiting hot Jupiters at low spectral resolution have had $n \approx 2$ \citep{stevenson16}.  

It should be emphasized how small these variations in transit depth are. Even for the best case scenario -- a large, low surface gravity planet with a hot atmosphere, orbiting a small star -- the amplitude of features is just $\sim0.1\%$. For an Earth analog, the expected amplitude is FIXME times smaller.  

A close cousin of the transmission spectroscopy method is emission spectroscopy.  


\section{Bibliography}


\begin{acknowledgement}
FIXME
\end{acknowledgement}

\bibliographystyle{spbasicHBexo}  %for bibtex
\bibliography{HBexoTemplatebib} %for bibtex-example

\end{document}
